\synctex=1

\documentclass{article}
\usepackage[utf8]{inputenc}




%%%%%%%%%%%%%%%%%%%%%%%%%%%%%%%%%%%%%%%%%%%%%%%%%%%%%
%%%%%%%%%%%%%%%%%%%%%%%%%%%%%%%%%%%%%%%%%%%%%%%%%%%%%
%%%%% Packages

\usepackage[utf8]{inputenc}
\usepackage[T1]{fontenc}
\usepackage{mathtools}
\usepackage{amssymb}
\usepackage{amsthm}
\usepackage{xspace}
\usepackage{float}
\usepackage{pgf}
\usepackage{tikz}
\usetikzlibrary{backgrounds,arrows,automata,positioning,calc,shadows,fit, patterns}
\usetikzlibrary{shapes.geometric}
\usepackage{array}
\usepackage{booktabs}
\usepackage{changepage}
\usepackage{hyperref}
\usepackage{enumitem}

%%%%%%%%%%%%%%%%%%%%%%%%%%%%%%%%%%%%%%%%%%%%%%%%%%%%%
%%%%%%%%%%%%%%%%%%%%%%%%%%%%%%%%%%%%%%%%%%%%%%%%%%%%%
%%%%% Environments

\newtheorem{theorem}{Theorem}[section]
\newtheorem{lemma}[theorem]{Lemma}
\newtheorem{corollary}[theorem]{Corollary}
\newtheorem{proposition}[theorem]{Proposition}
\newtheorem{claim}[theorem]{Claim}
\newtheorem*{metaproblem}{Meta Problem}
\newtheorem{realproblem}[theorem]{Problem}
\newtheorem{fact}[theorem]{Fact}

\newenvironment{proofsketch}{\noindent\emph{Proof sketch.}}{\hfill$\square$}

\theoremstyle{definition}
\newtheorem{definition}{Definition}
\newtheorem{remark}{Remark}
\newtheorem{example}{Example}

%%%%%%%%%%%%%%%%%%%%%%%%%%%%%%%%%%%%%%%%%%%%%%%%%%%%%
%%%%%%%%%%%%%%%%%%%%%%%%%%%%%%%%%%%%%%%%%%%%%%%%%%%%%
%%%%% Misc. Math

\newcommand{\cceq}{\mathop{::=}}
\newcommand{\myeq}[1]{\stackrel{\text{#1}}{\Leftrightarrow}}

%%%%%%%%%%%%%%%%%%%%%%%%%%%%%%%%%%%%%%%%%%%%%%%%%%%%%
%%%%%%%%%%%%%%%%%%%%%%%%%%%%%%%%%%%%%%%%%%%%%%%%%%%%%
%%%%% Greek Letters
\renewcommand{\epsilon}{\varepsilon}
\renewcommand{\phi}{\varphi}

%%%%%%%%%%%%%%%%%%%%%%%%%%%%%%%%%%%%%%%%%%%%%%%%%%%%%
%%%%%%%%%%%%%%%%%%%%%%%%%%%%%%%%%%%%%%%%%%%%%%%%%%%%%
%%%%% Basic Math

\newcommand{\pow}[1]{2^{#1}}
\newcommand{\nats}{\mathbb{N}}
\newcommand{\size}[1]{|#1|}
\newcommand{\set}[1]{\{ #1 \}}
\newcommand{\ap}[0]{\mathrm{AP}}
\newcommand{\subword}{\preceq}
\newcommand{\nsubword}{\npreceq}
\newcommand{\popartitions}[1]{\mathcal{POP}(#1)}

%%%%%%%%%%%%%%%%%%%%%%%%%%%%%%%%%%%%%%%%%%%%%%%%%%%%%
%%%%%%%%%%%%%%%%%%%%%%%%%%%%%%%%%%%%%%%%%%%%%%%%%%%%%
%%%%% LTL operators

\newcommand{\true}{\mathbf{tt}}
\newcommand{\false}{\mathbf{ff}}
\newcommand{\F}{{\mathbf{F\,}}}
\newcommand{\G}{{\mathbf{G\,}}}
\newcommand{\U}{{\mathbf{\,U\,}}}
\newcommand{\X}{{\mathbf{X\,}}}
\newcommand{\R}{{\mathbf{\,R\,}}}
\newcommand{\Exists}{{\mathbf{E\,}}}
\newcommand{\All}{{\mathbf{A\,}}}


%%%%%%%%%%%%%%%%%%%%%%%%%%%%%%%%%%%%%%%%%%%%%%%%%%%%%
%%%%%%%%%%%%%%%%%%%%%%%%%%%%%%%%%%%%%%%%%%%%%%%%%%%%%
%%%%% Abbrev. for Logics

\newcommand{\ltl}{{LTL}\xspace}
\newcommand{\ctl}{{CTL}\xspace}
\newcommand{\ctlstar}{{CTL$^*$}\xspace}
\newcommand{\fol}{{FO}$[<]$\xspace}
\newcommand{\fole}{{FO}$[<,\,E]$\xspace}
\newcommand{\logics}{$\Lambda$}
\newcommand{\hylogics}{Hyper$\Lambda$\xspace}
\newcommand{\msol}{{MSO}$[<]$\xspace}


\newcommand{\suffix}[2]{#1^{\geq #2}}
\newcommand{\prefix}[2]{#1^{\leq #2}}
\newcommand{\var}{\mathcal{V}}
\newcommand{\eqpoint}[1]{=_{#1}}
\newcommand{\subtree}[2]{#1_{\downarrow #2}}

%%%%%%%%%%%%%%%%%%%%%%%%%%%%%%%%%%%%%%%%%%%%%%%%%%%%%
%%%%%%%%%%%%%%%%%%%%%%%%%%%%%%%%%%%%%%%%%%%%%%%%%%%%%
%%%%% Trace names

\newcommand{\pit}{\pi\text{-trace}}
\newcommand{\taut}{\tau\text{-trace}}
\newcommand{\pits}{\pi\text{-traces}}
\newcommand{\tauts}{\tau\text{-traces}}

%%%%%%%%%%%%%%%%%%%%%%%%%%%%%%%%%%%%%%%%%%%%%%%%%%%%%
%%%%%%%%%%%%%%%%%%%%%%%%%%%%%%%%%%%%%%%%%%%%%%%%%%%%%
%%%%% Complexity

\newcommand{\poly}{\textsc{P}\xspace}
\newcommand{\np}{\textsc{NP}\xspace}
\newcommand{\conp}{\textsc{coNP}\xspace}
\newcommand{\sigmatwo}{$\Sigma_2^{\textsc{P}}$\xspace}
\newcommand{\pitwo}{$\Pi_2^{\textsc{P}}$\xspace}
\newcommand{\sigmathree}{$\Sigma_3^{\textsc{P}}$\xspace}
\newcommand{\pspace}{\textsc{PSpace}\xspace}
\newcommand{\expt}{\textsc{ExpTime}\xspace}
\newcommand{\nexpt}{\textsc{NExpTime}\xspace}
\newcommand{\exps}{\textsc{ExpSpace}\xspace}
\newcommand{\twoexpt}{\textsc{2ExpTime}\xspace}
\newcommand{\tower}{\textsc{Tower}\xspace}

%%%%%%%%%%%%%%%%%%%%%%%%%%%%%%%%%%%%%%%%%%%%%%%%%%%%%
%%%%%%%%%%%%%%%%%%%%%%%%%%%%%%%%%%%%%%%%%%%%%%%%%%%%%
%%%%% Trees
\newcommand{\aut}{\mathcal{A}}
\newcommand{\lang}[1]{\mathcal{L}(#1)}
\newcommand{\initmark}{I}
\newcommand{\kripke}{\mathcal{K}}
\newcommand{\src}[1]{src(#1)}
\newcommand{\tgt}[1]{tgt(#1)}
\newcommand{\runs}[1]{\mathcal{R}(#1)}
\newcommand{\traces}[1]{\mathcal{L}(#1)}
\newcommand{\trace}[1]{tr(#1)}
\newcommand{\game}{\mathcal{G}}
\newcommand{\val}[2]{val_{#1,#2}}
\newcommand{\valstrat}[3]{val^{#2,#3}_{#1}}
\newcommand{\trees}[1]{\mathcal{T}(#1)}
\newcommand{\subformulas}[1]{{Sub(#1)}}
\newcommand{\sons}[2]{Sons_{#2}(#1)}
\newcommand{\treeroot}[1]{root(#1)}
\newcommand{\ancestor}[1]{\preccurlyeq_{#1}}
\newcommand{\descendant}{\succcurlyeq}

%%%%%%%%%%%%%%%%%%%%%%%%%%%%%%%%%%%%%%%%%%%%%%%%%%%%%
%%%%%%%%%%%%%%%%%%%%%%%%%%%%%%%%%%%%%%%%%%%%%%%%%%%%%
%%%%% Probabilities

\newcommand{\distri}[1]{\mathcal{D}(#1)}
\newcommand{\proba}[1]{\mathbb{P}(#1)}

\usepackage{algpseudocode}
\usepackage[notion,quotation]{knowledge}

\input{kl}
\newif\ifappendix
\newrobustcmd\introinrestatable[1]{%
	\ifappendix%
	\kl{#1}%
	\else%
	\intro{#1}%
	\fi%
}

\newcommand{\corto}[1]{\todo[color=red!30]{\small #1}}
\newcommand{\cortoin}[1]{\todo[color=red!30,inline]{\small #1}}



\title{Population games on polynomially ambiguous automata}
\author{}
\date{}

\begin{document}

\maketitle

An automaton is \emph{very weak} if there exists a total order $\leq$ on its states such that all transitions $(s, a, t)$ are such that $s \leq t$.
In the first section we prove that deciding stochastic population games on such automata is PSPACE-hard.

An automaton is \emph{polynomially ambiguous} if there exists a polynomial function $f : \nats \to \nats$ such that for all $w \in \Sigma^*$, $w$ labels at most $f(\size{w})$ accepting runs. 
If we assume that automata are trimmed (that is, all states are reachable from an initial state and can reach a final one), this is equivalent to saying that for all states $s$ and words $w$ there is at most one path from $s$ to itself reading $w$.
In the second section we prove that we can decide stochastic population games on such automata in PSPACE.

As very weak automata are polynomially ambiguous, we obtain PSPACE-completeness of the problem for both those classes.

\section{Lower bound}

Here is a PSPACE lower bound: 

We reduce the emptiness problem for Zielonka automata. Those are automata over a set $Proc$ of processes, with a family of sets of states $(S_p)_{p \in Proc}$, another of initial states $(init_p)_{p \in Proc}$, a set of global final states $F \subseteq \Pi_{p\in Proc} S_p$ and transitions of the form $(s_{p_1}, \ldots, s_{p_k}) \xrightarrow{a} (s'_{p_1}, \ldots, s'_{p_k})$, with $p_1, \ldots, p_k \in Proc$.

We set up the following game: 

It consists of two parts: One will be in charge of simulating the effect of the sequence of actions chosen by the player on the configuration of the automaton.

The other is a timer (in the form of a binary counter) ensuring that at most  $M = \Pi_{p\in Proc} |S_p|$ are played.

We first describe the automaton simulation: 

We have a reservoir of tokens, and one place for each state of each process.
The actions are the transitions of the automaton. The effect of an action $(s_{p_1}, \ldots, s_{p_k}) \xrightarrow{a} (s'_{p_1}, \ldots, s'_{p_k})$ is to send all tokens from the $s_{p_i}$ to the target place, kill all tokens in the other states of the $p_i$, split the tokens of the reservoir between itself and the $s'_{p_i}$ and loop on all states of all other processes.

For all $(f_p)_{p \in Proc} \in F$ we have an action that sends all tokens from the $f_p$ and the reservoir to the target place, and kills all other places. 

Observe that if the automaton accepts a word, then it accepts one of size less than $M$, and we can send all tokens to the target place by reading an accepting sequence of that word, then the action corresponding to the final configuration reached.  

Now suppose we can send all tokens to the target with $M$ actions, then by starting with a large enough number of tokens, there is a positive probability that all splits of the reservoir send at least one token in all its successors. As a result, a winning run in $M$ moves has to simulate a run of the automaton. 
In order to be winning, it needs to end with some $f \in F$, hence the corresponding run is accepting. 

Now we describe the other part of the game, playing in parallel.
We set $n$ as the integer part of $log_2(M)+1$. We have $2n$ places, of the form $(j,b)$ with $0 \leq j \leq  n$ and $b \in \set{0,1}$, plus a reservoir.
An action is a number $0\leq i < n$. Its effect is to send all tokens of $(i', 0)$ to $(i', 1)$ and kill those in $(i',1)$ for all $i'\leq i$. It also kills the tokens in $(i,0)$, sends the tokens of $(i,1)$ to the target and splits the tokens of the reservoir between itself and $(i,0)$. It leaves the rest untouched.

For each $i$ there is also an action $end_i$ which kills all tokens in $(i,0)$ and sends all others to the target state.

Now we describe the entire game: We take the union of the two parts described above, plus initial and final places. At first the tokens of the initial place are split between the reservoirs of the two parts, as well as the places $(init_p)_{p \in Proc}$ and $(i,1)$.

Then the rest of the actions are pairs $(x,y)$ with $x$ an action of the automaton part and $y$ an action of the counter part.

Their effects are applied independently on each part of the graph.

Suppose there is an accepting run of the automaton, then there is one of length less than $M < 2^n$, hence we can play the corresponding actions in the automaton part while decreasing the counter in the other part at each step.
In the end the counter has not reached $0$, hence there is an $i$ such that $(i,0)$ is empty. Meanwhile, the automaton has reached a final configuration $f \in F$, hence we can play $(end_i, f)$ and win.

Now suppose there is a winning strategy. We can assume that are numerous enough so that during the first $M$ steps of the play every reached enabled transition is taken by some tokens at all times.

As a consequence, such a play cannot last more than $M$ steps, as otherwise the binary counter would reach $0$, with tokens in all $(i,0)$ places, meaning that no action can be played without killing some of them.
Hence in the automaton part we had to follow an accepting run.

\section{Upper bound}

Let $\aut = (S, \Sigma, \Delta, init, fin)$ be a \textbf{complete} polynomially ambiguous automaton with one initial and one final state (which we can assume to be a sink state). We can assume completeness as adding a sink state to an incomplete automaton preserves the polynomially ambiguous property.

We define the support of a configuration $C \in \nats^S$ as $supp(C) = \set{s \in S \mid C(s) > 0}$.

A strategy $\sigma : \nats^S \to \Sigma$ for the associated population game is \intro{simple} if for all $C, C' \in \nats^S$, if $supp(C) = supp(C')$ then $\sigma(C) = \sigma(C')$.

A \intro{split} is a pair $(C,a)$ such that $C$ is a configuration and $a$ an action such that there exists $s \in supp(C)$ such that $\Delta(s,a) > 1$.

A support is \emph{winning} if the configurations obtained by putting arbitrary numbers in its states are all winning.

\begin{lemma}
	\label{lem:simple-strat}
	There exists a "simple" strategy guaranteeing that:
	\begin{itemize}
		\item The target configuration is eventually reached from every configuration with a winning support.
		
		\item There is a bound $M$ such that for all $N$, from every configuration with $N$ tokens and winning support, there is a positive probability $p$ that a token moves to a different SCC of the automaton within the next $M$ steps while following this strategy.
	\end{itemize} 
\end{lemma}

\begin{proof}
	Let $sup$ be a winning support different from $\set{fin}$, we show that there exists a word $w$ such that by reading $w$ from a configuration of $sup$ we are guaranteed to stay in configurations of winning supports and there is a path labelled by $w$ from one of the non-empty states of the configuration to a state in a different SCC.
	
	Suppose it is not the case.
	If a word $w$ is such that $\Delta(s, w)$ intersects an SCC different from the one of $s$ then $\Delta(sup, w)$ is not a winning support. Otherwise $\Delta(s, w)$ is included in the SCC of $s$, and $w$ can label at most $\size{S}$ paths from $s$, as otherwise there would be some $t$ in the same SCC as $s$ such that $w$ labels two paths from $s$ to $t$. As they are in the same SCC, there is a path from $t$ to $s$ labelled by some word $v$, and thus $wv$ labels two paths from $s$ to itself, contradicting the assumption that $\aut$ is polynomially ambiguous.
	
	As a consequence, all words read from $sup$ either lead to a non-winning support or label at most $\size{S}^2$ paths from states of $sup$ (at most $\size{S}$ from each state).
	
	Let $N$ be such that for all non-winning support $sup'$, $(sup')^N$ is not a winning configuration.
	
	We can start with a number of tokens high enough in each state of $sup$ so that there is a positive probability that at least $N$ tokens spread to each successor of each non-empty state during the first $\size{S}^2$ "split" actions.
	
	Thus the actions selected by $\sigma$ up to the $\size{S}^2$th "split" have to form a word $w$ such that $\Delta(s, w)$ is included in the SCC of $s$ for all $s \in sup$, otherwise the configuration obtained would be higher than $(sup')^N$ with $sup'$ a non-winning support, and thus $\sigma$ would not be winning.
	
	As a result, the $\size{S}^2$th split action can never happen, as it would get to a losing configuration with positive probability.
	
	Thus $\sigma$ has to keep playing forever without ever getting all tokens to $fin$, as $sup = fin$ and tokens cannot leave their SCCs. Hence $\sigma$ is losing, yielding a contradiction.

	Then we construct our "simple" strategy as follows: for all $sup \in \pow{S}$, it picks an arbitrary word whose prefixes all lead to winning supports, and which labels a path from one of the states of $sup$ to a state of a different SCC, of minimal length. Our strategy plays the first letter of that word.
	
	Let $m$ be the maximal length of one of those words over all winning supports.
	At all times there is a probability at least $\frac{1}{\size{S}^m}$ that a token reaches a different SCC within the next $m$ steps, and we always stay in the set of winning supports. As a result, eventually all tokens must reach $fin$.
	
	Thus our strategy is winning. 
	
%	Consider an infinite run $C_0 \xrightarrow{a_1} C_1 \xrightarrow{a_2} \cdots$ respecting $\sigma$ and such that the first $\size{S}^2+1$ "split" actions are such that tokens spread to all successors of all non-empty states (this is possible with enough tokens).
%	For all $i$ let $w_i = a_1 \cdots a_i$. By assumption for all $i$ 
%	
%	We consider a strategy $\sigma : \nats^S \to \Sigma$, which is positional and winning from every winning configuration.
%	
%	For all $w \in \Sigma^*$, there are at most 
%	
%	
%	
%	We construct our simple strategy as follows: 
%	
%	We have a set $\mathcal{C} \subseteq \pow{S}$ of supports, initially empty.
%	
%	We pick a winning support $sup_0$ outside of $\mathcal{C}$. If there is none, we stop.
%	We also set $I_0 = \nats$.
%	
%	For all $i \in \nats$, we define inductively $a_{i+1}$ such that $\sigma(sup_{i}^N) = a_{i+1}$ for infinitely many $N \in I_i$, where $sup_{i}^N$ is the configuration obtained by putting $N$ tokens in each state of $sup_i$. We also define the infinite set $I_{i+1} = \set{N \in I_i \mid \sigma(sup_i^N)=a_{i+1}}$ and $sup_{i+1} = \Delta(sup_i, a_{i+1})$.
%	
%	Clearly all $sup_i$ are winning supports: $sup_0$ is a winning support, and for all $i$, if $sup_i$ is a winning support, then for infinitely many $N$, $sup_i^N$ is a winning configuration. Let $N \in \nats$, there exists $K \in I_{i+1}$ large enough such that by playing $a_{i+1}$ from $sup_i^K$ we have a positive probability of getting a configuration with at least $N$ tokens in all states of $sup_{i+1}$.
% 	The set of winning configurations is downwards-closed, hence $sup_{i+1}^N$ is winning for all $N$.
% 	 
%	Then we distinguish cases: 
%	
%	\paragraph{If $supp_{i} \notin \mathcal{C}$ for all $i$:} Then there exists $0 \leq i_1 < i_2 \leq \pow{\size{S}}$ such that $sup_i = sup_j$.
%	
%	We prove that either one of the actions $a_i$, $i_1 < i \leq i_2$ is a "split" or a "merge", or one of the supports $sup_i$, $i_1 \leq i < i_2$, is $\set{fin}$:
%	suppose we are not in the first case, then we pick some $N \in I_{i_2}$. As $(I_i)_{i \in \nats}$ is a decreasing sequence, we have $N \in I_{i}$ for all $i_1 \leq i < i_2$. 
%	As no action $a_i$, $i_1 \leq i \leq i_2$, is a "split" or a "merge" on $sup^N_{i-1}$, for all $0 \leq j \leq i_2 - i_1$, after playing $j$ moves from $sup^N_{i_1}$ according to $\sigma$, we obtain the configuration $sup_{i_1 + j}^N$, as each group of $N$ tokens just moves without ever splitting or encountering another one.
%	Hence after $i_2 - i_1$ steps, we obtain $sup_{i_2}^N = sup_{i_1}^N$, meaning that $\sigma$ only visits configurations $\set{sup_i^N \mid i_1 \leq i \leq i_2}$ from then on. As $\sigma$ is winning, one of those configurations has to be $\set{fin}^N$, hence we have $sup_i = \set{fin}$ for some $i_1 \leq i < i_2$.
%	
%	For all $sup_i$, $i \in \nats$, we set $\sigma'(sup_i) = a_{j+1}$, with $j = min \set{i' \in \nats \mid sup_{i'} = sup_i}$.
%	
%	\paragraph{If $sup_{i} \in \mathcal{C}$ for some $i$:} We pick the minimal such $i$, and for all $j < i$ we set $\sigma'(sup_j) = a_j$.
%	
%	
%	We iterate this construction until $\mathcal{C}$ is the set of all winning supports.
%	We obtain a strategy that stays in the configurations with winning supports, and guarantees that from every such configuration we eventually do a "split", a "merge", or reach the target configuration. 
%	
%	All that is left to do is to prove that this strategy is winning from all configurations with winning supports.
%	
%	We can do at most $\size{S}$ merges without splitting, as a merge decreases the number of states containing tokens.
%	
%	Say we may have a run with more than $\size{S}^2$ splits and such that at no point may any token leave its current connected component. Let $a_1 \cdots a_n$ be the letters read along that run.
%	For all $s \in S$, 
%	
%	
\end{proof}

We now consider flows, which are just elements of $\pow{S^2}$.
Given a flow $f$, we define $pre(f) = \set{s \in S \mid \exists t \in S, (s,t) \in f}$ and $post(f) = \set{t \in S \mid \exists s \in S, (s,t) \in f}$
We equip then with a composition operation:

If $post(f_1) = pre(f_2)$, we define their composition as \[f_1 \cdot f_2 = \set{(s, t) \mid \exists s' \in S, (s, s') \in f_1 \land (s', t) \in f_2}\]

We now define accelerations: 

If $f\cdot f = f$ (i.e. $f$ is idempotent), then its acceleration is \[f^\# = \set{(s,t) \mid (t, t) \in f \land (t,t') \notin f \text{ for all } t' \neq t \land (s, t) \in f }\]

We now consider the $\#$-expressions defined by the grammar \[E ::= a (a \in \Sigma) \mid EE \mid E^\#\]

To all $\#$-expressions $E$ we associate a flow $f(E)$ defined inductively as follows:

\begin{itemize}
	\item $f(a) = \set{(s,t) \mid t \in \Delta(s,a)}$ for all $a \in \Sigma$.
	
	\item $f(E_1 E_2) = f(E_1)\cdot f(E_2)$ (if the product is defined) for all $E_1, E_2$ 
	
	\item $f(E^\#) = f(E)^\#$ (if the acceleration is defined) for all $E$ 
\end{itemize}

\begin{lemma}
	\label{lem:post-idem}
	For all idempotent $f \in \pow{S^2}$, $f^\# \subseteq f$ and if $post(f^\#) = post(f)$ then $f = f^\#$.  
\end{lemma}

\begin{proof}
	The fact that $f^\# \subseteq f$ follows directly from the definition.
	Suppose $post(f^\#) = post(f)$, and let $(s,t) \in f$. 
	As $t \in post(f^\#)$, we have that $(t,t) \in f$ and $(t,t') \notin f$ for all $t'\neq t$. As a result, $(s,t) \in f^\#$. 
	Thus $f = f^\#$.
\end{proof}

\begin{lemma}
	\label{lem:word-exp}
	For all $E$, there exists $w \in \Sigma^*$ such that $f(E) \subseteq \Delta(w)$.
	Furthermore, for all $s \in S$ there exists $t \in S$ such that $(s,t) \in f(E)$.
\end{lemma}

\begin{proof}
	We prove this by induction on the number of $\#$ in $E$.
	If $E$ contains no $\#$ then $E \in \Sigma^*$ and $f(E) = \Delta(E)$.
	If $E$ is of the form $E_1(E_2)^\# E_3$ then by induction hypothesis there exist $w_1, w_2, w_3$ with $f(E_i) \subseteq \Delta(w_i)$ for each $i$.
	As a consequence $f(E_2^\#) \subseteq f(E_2) \subseteq \Delta(w_2)$, and $f(E) \subseteq \Delta(w_1w_2w_3)$.
	
	For the second part, we proceed by induction: as our automaton is complete, all $f(a)$ map all state to at least one other state.
	Clearly this property is stable by composition and acceleration, proving the lemma.
\end{proof}

\begin{lemma}
	\label{lem:stab-property}
	For all $E_1, E_2$ such that $f(E_1E_2)$ and $f(E_1E_2)$ are idempotent, $f((E_1E_2)^\#) = f(E_1(E_2E_1)^\# E_2)$.
\end{lemma}

\begin{proof}
	We set $e_1 = f(E_1)$ and $e_2 = f(E_2)$.
	
	We unfold the definitions: Let $(s,t) \in (e_1 \cdot e_2)^\#$, we have that $(s,t), (t,t) \in e_1 \cdot e_2$ and $(t,t') \notin e_1 \cdot e_2$ for all $t' \neq t$.
	
	Thus there exist $s', s''$ such that $(s,s'), (t, s'') \in e_1$ and $(s',t), (s'',t) \in e_2$.  As a consequence, $(s', s''), (s'', s'') \in e_2 \cdot e_1$. Further, let $t' \neq s''$, suppose we have $(s'', t') \in e_2\cdot e_1$ then as our automaton is complete there exists $t''$ such that $(t', t'') \in e_2$ and thus $(t, t'') \in e1\cdot e_2 \cdot e_1 \cdot e_2$.
	As $(t, u) \notin e_1 \cdot e_2$ for all $u \neq t$, we must have $t=t''$.
	
	However this implies, by Lemma~\ref{lem:word-exp}, that there exist $w_1, w_2$ such that $(t, s'') \in \Delta(w_1)$, $(s'', t), (s'', s'') \in \Delta(w_2w_1)$ and $(s'',t) \in \Delta(w_2)$. As a consequence, $(w_1w_2)^2$ labels two paths from $t$ to itself, contradicting the fact that $\aut$ is polynomially ambiguous.
	
	As a result, for all $t' \neq s''$, $(s'',t') \notin e_2\cdot e_1$. By definition, this means that $(s', s'') \in (e_2 \cdot e_1)^\#$, and thus that $(s,t) \in e_1 (e_2 \cdot e_1)^\# e_2$.
	We have shown that $(e_1\cdot e_2)^\# \subseteq e_1 (e_2 \cdot e_1)^\# e_2$.
	
	We now prove the other direction. Let $(s,t) \in e_1 (e_2 \cdot e_1)^\# e_2$, there exist $s', t'$ such that $(s, s') \in e_1$, $(t',t) \in e_2$ and $(s', t') \in (e_2 \cdot e_1)^\#$.
	By definition, $(s', t') \in e_2\cdot e_1$ and thus $(s,t) \in (e_1\cdot e_2)^2 = e_1 \cdot e_2$. 
	Also, $(t', t) \in e_2$. By Lemma~\ref{lem:word-exp}, there exists $t''$ such that $(t,t'') \in e_1$, and thus $(t', t'') \in e_2 e_1$. As $(s', t') \in (e_2 e_1)^\#$, $t'' = t'$.
	
	Thus $e_1(t) = \set{t'}$. As a consequence, $(t,t) \in e_1e_2$. Now suppose there exists $t'' \neq t$ such that $(t', t'') \in e_2$, then by Lemma~\ref{lem:word-exp} there exists $t'''$ such that $(t'', t''') \in e_1$, and thus $(t', t''') \in e_2 e_1$, meaning that $t''' = t'$.
	
	By Lemma~\ref{lem:word-exp}, there exist words $w_1, w_2$ with $e_1 \subseteq \Delta(w_1)$ and $e_2 \subseteq \Delta(w_2)$. Hence there are paths labelled by $w_2$ from $t'$ to $t$ and $t''$, and by $w_1$ from $t$ and $t''$ to $t'$. This contradicts the fact that $\aut$ is polynomially ambiguous as there would then be two paths from $t'$ to itself.
	
	 As a result, $e2(t') = \set{t}$, and thus $e_1 e_2 (t) = \set{t}$.
	 Hence $(s,t) \in (e_1 e_2)^\#$.
	 
	 This concludes our proof.
\end{proof}

\begin{lemma}
	\label{lem:small-depth}
	For all $E$ there exists an equivalent expression $F$ with at most $\size{S}$ nested $\#$.
\end{lemma}

\begin{proof}
	First of all we eliminate useless $\#$: while $E$ has a subexpression $G^\#$ with $f(G) = f(G^\#)$, we replace $G^\#$ with $G$. This terminates as the size of $E$ is decreased at each iteration.
	
	Then we show that we can transform $E$ so that for all subexpressions $G^\#$ where $G$ has $k$ nested $\#$, $\size{post(f(G^\#))} \leq \size{S} - k -1$.
	
	Let $k \in \nats$, suppose we already have this property for all $k' < k$.
	We consider each subformula of the form $G^\#$ with $G$ having $k$ nested $\#$.
	If $k=0$ then if we had $\size{post(f(G^\#))} = \size{S}$, then necessarily we would have $f(G^\#) = \set{(s,s) \mid s \in S}$ and thus $post(f(G^\#)) = S = post(f(G))$, hence $f(G) = f(G^\#)$ by Lemma~\ref{lem:post-idem}, which is impossible as we eliminated those subexpressions.
	
	If $k > 0$, then $G = E_1 (E_2)^\# E_3$ with $E_2$ of $\#$ nesting depth $k-1$. Hence $\size{post(f(E_2^\#))} \leq \size{S} - k$. 
	Furthermore, $f(G^\#) = f(E_1(E_2)^\# (E_3 E_1(E_2)^\#)^\# E_3)$. 
	If we had $f((E_3 E_1(E_2)^\#)^\#) = f(E_3 E_1(E_2)^\#)$, then using Lemma~\ref{lem:stab-property} we would have $f(G^\#) = f(E_1(E_2)^\# (E_3 E_1(E_2)^\#)^\# E_3) = f(E_1(E_2)^\# E_3 E_1(E_2)^\# E_3) = f(G)^2 = f(G)$, a contradiction.
	
	Hence $f((E_3 E_1(E_2)^\#)^\#) \neq f(E_3 E_1(E_2)^\#)$, therefore by Lemma~\ref{lem:post-idem} $\size{post(f((E_3 E_1(E_2)^\#)^\#))} < \size{post(f(E_3 E_1(E_2)^\#))} = \size{post(f((E_2)^\#))} \leq \size{S} -k$.
	
	As a result, we can replace $G^\#$ with the expression $E_1(E_2)^\# (E_3 E_1(E_2)^\#)^\# E_3$ (equivalent by Lemma~\ref{lem:stab-property}), of equal $\#$ nesting depth, which respects the desired property.
\end{proof}

\begin{lemma}
	\label{lem:exp-to-strat}
	For all $E$ there exists a strategy $\sigma$ that ensures that from any configuration $C$ we almost surely eventually reach a configuration such that
	all tokens initially in a state $s$ are in a state $t$ such that $(s,t) \in f(E)$ and
	for all $(s,t) \in f(E)$ every token initially in $s$ has a positive probability to end up in $t$.
	In particular if there exists $E$ such that $f(E)(init) =\set{fin}$, then there exists a winning strategy.
\end{lemma}

\begin{proof}
	We prove this by induction on $E$.
	
	If $E$ is a letter $a$, then the strategy that simply reads $a$ and stops works.
	
	If $E = E_1 E_2$, let $sup \subseteq S$, by induction hypothesis we have strategies $\sigma_1$ and $\sigma_2$ enforcing the flows $f(E_1)$ and $f(E_2)$.
	We apply $\sigma_1$ until we reach a configuration as described in the lemma.
	Then we apply $\sigma_2$.
	
	After applying $\sigma_1$ all tokens initially in a state $s$ are in states of $f(E_1)(s)$, thus in the end they are all in states of $f(E_1E_2)(s)$.
	
	Let $(s,t) \in f(E_1E_2)$, there exist $s' \in f(E_1)(s)$ such that $t \in f(E_2) (s')$. All tokens initially in $s$ have a positive probability to end up in $s'$ after applying $\sigma_1$, and from there a positive probability to reach $t$ in the end.
	
	If $E= F^\#$ then by induction hypothesis there is a strategy $\sigma_F$ enforcing the flow $f(F)$.
	
	We apply this strategy repeatedly.
	As $F$ is an idempotent, the relation $f(F)$ is transitive.
	Therefore by applying $\sigma_F$ some number of times a token initially in $s$ can only reach states that are in $f(F)(s)$.
	
	Let $s \in S$, if $s \notin f(F)(s)$ then after applying $\sigma_F$ some number of times tokens that were initially in $s$ have to be in other states, therefore $s$ is empty.
	
	If $s \in f(F)(s)$ and there exists $t \neq s$ such that $(s,t) \in f(F)$, then $(t, s) \notin f(F)$, as by Lemma~\ref{lem:word-exp} there exists a word $w$ such that $f(F) \subseteq \Delta(w)$ and thus there would be two paths labelled by $w^2$ from $s$ to itself, contradicting the fact that the automaton is polynomially ambiguous.
	
	As a result, after each iteration of $\sigma_F$ tokens that were in $s$ have a positive probability to end up in $t$ and never come back to $s$ after later iterations of $\sigma_F$. Hence we can iterate $\sigma_F$ enough times to empty $s$ entirely.
	
	We define $\sigma_E$ as the strategy iterating $\sigma_F$ until all states $s$ such that $f(F)(s) \neq \set{s}$ are empty, and at least once.
	Let $s,t \in S$, if $(s,t) \notin f(F)$ then no token initially in $s$ can be transferred to $t$ by applying $\sigma_E$ (as $f(F)$ is transitive).
	
	Furthermore, if $f(F)(t) \neq \set{t}$ then after applying $\sigma_E$ no token can be in $t$. By contrast, if $f(F)(t) = \set{t}$ and $(s,t) \in F$ then as we apply $\sigma_F$ at least once all tokens in $s$ have a positive probability to end up in $t$ on the first application of $\sigma_F$, and will then stay in $t$ forever if they did reach it. Hence all tokens in $s$ have a positive probability to reach $t$ after applying $\sigma_E$.
	
	This concludes our induction. 
\end{proof}

\begin{lemma}
	\label{lem:strat-to-exp}
	If there exists a winning strategy then there is an expression $E$ such that $f(E)(init) =\set{fin}$.
\end{lemma}

\begin{proof}
	Suppose there is a winning strategy, then there is one $\sigma$ satisfying the properties described in Lemma~\ref{lem:simple-strat}. 
	Let $M \in \nats$ be such that for all flow $f$, $f^M$ is idempotent.
	
	In order to construct an expression $E$ such that $f(E)(init) =\set{fin}$, we start in the initial support $\set{init}$ with $E = \epsilon$, and we proceed as follows:
	if we are in a support $sup$ we never saw before, we set $E := E\sigma(sup)$ and $E_{sup} = E$ and go to support $sup' = \Delta(\sigma(sup))(sup)$. As $\sigma$ wins with probability one from any configuration of winning support, if $sup$ is a winning support then so is $sup'$.
	If we are in a support $sup$ that we saw before, let $F$ be such that $E = E_{sup} F$. We set $E : = E (F^M)^\#$ and go to $sup' = f((F^M)^\#)(sup)$.
	As $f((F^M)^\#) \subseteq f(F)$, $sup' \subseteq sup$ thus if $sup$ is a winning support then so is $sup'$.
	
	Note that as $\set{init}$ is a winning support, all the supports we encounter are winning.
	Also, it is clear from the definition that at all times, if $sup$ is the current support and $E$ the current expression, then $f(E)(\set{init}) = sup$ and for all $sup'$ such that $E_{sup'}$ is defined, there exists $F$ such that $E = E_{sup'} F$ and $f(F)(sup') = sup$.
	
	The challenge now is to show that we eventually encounter the configuration $\set{fin}$.
	To do so, we prove that we can encounter each support $sup$ at most $\size{S}$ times beforehand. More precisely, we prove that for all $n\geq 2$, if we encounter $sup$ for the $n$th time, then in the next step we obtain a support $sup'$ of size at most $\size{sup}-n+1$.
	
	For $n = 2$, suppose by contradiction that $sup' = sup$. Let $E$ be the current expression when we encounter $sup$ for the second time and $F$ be such that $E = E_{sup} F$. If $f((F^M)^\#) \cap sup = \set{(s,s) \mid s\in sup}$ then $f(F^M) = f((F^M)^\#)$. If $F$ does not contain any $\#$ then this contradicts the fact that $\sigma$ is winning as it shows that from $sup$ the strategy enters a loop without making any splits and without finding support $\set{fin}$.
	 
	Let $sup = sup_1, sup_2, \ldots, sup_m, sup_{m+1}, \ldots, sup_{m+k}$ and $a_1, \ldots, a_{m+k-1}$ be such that $sup_m = sup_{m+k}$ and for all $i$, $\sigma(sup_i) = a_i$ and $\Delta(sup_i, a_i) = sup_{i+1}$.
	
	As $\sigma$ is winning, one of those $A-i$ has to send some state $s \in sup_i$ to one of a different SCC or reach $\set{fin}$.
	When we get to $sup$ for the second time, we must have taken all those actions at some point, as the step after first encountering $sup_i$ is executing $a_i$.
	
	Therefore one of those actions sent a state to one of a lower SCC, meaning that $\size{f(F)(s)}\geq 2$ for some $s \in sup$.
	Hence $s \notin f((F^M)^\#)(sup)$ and thus $sup' \subsetneq sup$.
	
	Now suppose $n\geq 3$.
	When we see $sup$ for the $n$th time we have an expression of the form $E = E_{sup} F (F^M)^\# G$ where $f(F) (sup) = sup$, with $E_{sup} F$ the expression we have when we encounter $sup$ for the $n-1$th time.
	
	Let $sup_{n-1} = f((F^M)^\#) (sup)$ and $sup_{n} = f((((F^M)^\# G)^M)^\#) (sup)$
	
	For all $s \in sup_n$, $f(((F^M)^\# G)^M)(s) = \set{s}$, thus there exists an unique $t$ such that $f(((F^M)^\# G)^{M-2} (F^M)^\# )(s) = \set{t}$ and $f(G(F^M)^\# G)(t) = \set{s}$.
	
	However,as $f(((F^M)^\# G)^{M-2} (F^M)^\#) (sup) = f((F^M)^\#)(sup) = sup_{n-1}$, by induction hypothesis this set contains at most $\size{sup}-n+2$. Furthermore by the previous argument some state must be sent to a lower component in $(F^M)^\# G$, hence we cannot have all of these $t$ sent to a single state and obtain $sup$ in the end.
	
	Hence $sup_n \subsetneq sup_{n-1}$, proving our result.
	
	As a consequence, we can encounter each support at most $\size{S}$ times before reaching $\set{fin}$.
\end{proof}

\begin{proposition}
	The random population control problem is PSPACE-complete for polynomially ambiguous automata.
\end{proposition}

\begin{proof}
	By Lemmas~\ref{lem:exp-to-strat} and~\ref{lem:strat-to-exp}, there exists a winning strategy if and only if there exists an expression $E$ such that $f(E)(init) = \set{fin}$.
	We will therefore look for such an expression. 
	
	Furthermore, by Lemma~\ref{lem:small-depth} we only have to look for expressions of $\#$-depth at most $\size{S}$. This allows us to design the following NPSPACE algorithm: 
	
	\begin{algorithmic}
		\State Input: a flow $fl$, an integer $n$
		\State Output: an expression $E$ of $\#$-depth at most $n$ such that $f(E) = fl$, or $\bot$ if it does not exist. 
		\Function{SearchExp}{fl,n} 
		\If{$n<0$}
			\Return $\bot$
		\EndIf
		\State $E = \epsilon$
		\While{$f(E) \neq fl$}
			\State Guess boolean $IsLetter$
			\If{$IsLetter$}
				\State Guess letter $a \in \Sigma$ 
				\State $E \gets Ea$ 
			\Else
				\State Guess an idempotent flow $flow \subseteq \pow{S^2}$
				\State $F \gets \Call{SearchExp}{flow, n-1}$
				\If{$F = \bot$}
					\Return $\bot$
				\EndIf 
				\State $E \gets EF$
			\EndIf
		\EndWhile
		\Return $E$
		\EndFunction
	\end{algorithmic}

We decide the problem by guessing a flow $fl$ such that $fl(init) = \set{fin}$ and executing $\sc{SearchExp}(fl, \size{S})$.
\end{proof}



\end{document}