\synctex=1

\documentclass{article}
\usepackage[utf8]{inputenc}



%%%%%%%%%%%%%%%%%%%%%%%%%%%%%%%%%%%%%%%%%%%%%%%%%%%%%
%%%%%%%%%%%%%%%%%%%%%%%%%%%%%%%%%%%%%%%%%%%%%%%%%%%%%
%%%%% Packages

\usepackage[utf8]{inputenc}
\usepackage[T1]{fontenc}
\usepackage{mathtools}
\usepackage{amssymb}
\usepackage{amsthm}
\usepackage{xspace}
\usepackage{float}
\usepackage{pgf}
\usepackage{tikz}
\usetikzlibrary{backgrounds,arrows,automata,positioning,calc,shadows,fit, patterns}
\usetikzlibrary{shapes.geometric}
\usepackage{array}
\usepackage{booktabs}
\usepackage{changepage}
\usepackage{hyperref}
\usepackage{enumitem}

%%%%%%%%%%%%%%%%%%%%%%%%%%%%%%%%%%%%%%%%%%%%%%%%%%%%%
%%%%%%%%%%%%%%%%%%%%%%%%%%%%%%%%%%%%%%%%%%%%%%%%%%%%%
%%%%% Environments

\newtheorem{theorem}{Theorem}[section]
\newtheorem{lemma}[theorem]{Lemma}
\newtheorem{corollary}[theorem]{Corollary}
\newtheorem{proposition}[theorem]{Proposition}
\newtheorem{claim}[theorem]{Claim}
\newtheorem*{metaproblem}{Meta Problem}
\newtheorem{realproblem}[theorem]{Problem}
\newtheorem{fact}[theorem]{Fact}

\newenvironment{proofsketch}{\noindent\emph{Proof sketch.}}{\hfill$\square$}

\theoremstyle{definition}
\newtheorem{definition}{Definition}
\newtheorem{remark}{Remark}
\newtheorem{example}{Example}

%%%%%%%%%%%%%%%%%%%%%%%%%%%%%%%%%%%%%%%%%%%%%%%%%%%%%
%%%%%%%%%%%%%%%%%%%%%%%%%%%%%%%%%%%%%%%%%%%%%%%%%%%%%
%%%%% Misc. Math

\newcommand{\cceq}{\mathop{::=}}
\newcommand{\myeq}[1]{\stackrel{\text{#1}}{\Leftrightarrow}}

%%%%%%%%%%%%%%%%%%%%%%%%%%%%%%%%%%%%%%%%%%%%%%%%%%%%%
%%%%%%%%%%%%%%%%%%%%%%%%%%%%%%%%%%%%%%%%%%%%%%%%%%%%%
%%%%% Greek Letters
\renewcommand{\epsilon}{\varepsilon}
\renewcommand{\phi}{\varphi}

%%%%%%%%%%%%%%%%%%%%%%%%%%%%%%%%%%%%%%%%%%%%%%%%%%%%%
%%%%%%%%%%%%%%%%%%%%%%%%%%%%%%%%%%%%%%%%%%%%%%%%%%%%%
%%%%% Basic Math

\newcommand{\pow}[1]{2^{#1}}
\newcommand{\nats}{\mathbb{N}}
\newcommand{\size}[1]{|#1|}
\newcommand{\set}[1]{\{ #1 \}}
\newcommand{\ap}[0]{\mathrm{AP}}
\newcommand{\subword}{\preceq}
\newcommand{\nsubword}{\npreceq}
\newcommand{\popartitions}[1]{\mathcal{POP}(#1)}

%%%%%%%%%%%%%%%%%%%%%%%%%%%%%%%%%%%%%%%%%%%%%%%%%%%%%
%%%%%%%%%%%%%%%%%%%%%%%%%%%%%%%%%%%%%%%%%%%%%%%%%%%%%
%%%%% LTL operators

\newcommand{\true}{\mathbf{tt}}
\newcommand{\false}{\mathbf{ff}}
\newcommand{\F}{{\mathbf{F\,}}}
\newcommand{\G}{{\mathbf{G\,}}}
\newcommand{\U}{{\mathbf{\,U\,}}}
\newcommand{\X}{{\mathbf{X\,}}}
\newcommand{\R}{{\mathbf{\,R\,}}}
\newcommand{\Exists}{{\mathbf{E\,}}}
\newcommand{\All}{{\mathbf{A\,}}}


%%%%%%%%%%%%%%%%%%%%%%%%%%%%%%%%%%%%%%%%%%%%%%%%%%%%%
%%%%%%%%%%%%%%%%%%%%%%%%%%%%%%%%%%%%%%%%%%%%%%%%%%%%%
%%%%% Abbrev. for Logics

\newcommand{\ltl}{{LTL}\xspace}
\newcommand{\ctl}{{CTL}\xspace}
\newcommand{\ctlstar}{{CTL$^*$}\xspace}
\newcommand{\fol}{{FO}$[<]$\xspace}
\newcommand{\fole}{{FO}$[<,\,E]$\xspace}
\newcommand{\logics}{$\Lambda$}
\newcommand{\hylogics}{Hyper$\Lambda$\xspace}
\newcommand{\msol}{{MSO}$[<]$\xspace}


\newcommand{\suffix}[2]{#1^{\geq #2}}
\newcommand{\prefix}[2]{#1^{\leq #2}}
\newcommand{\var}{\mathcal{V}}
\newcommand{\eqpoint}[1]{=_{#1}}
\newcommand{\subtree}[2]{#1_{\downarrow #2}}

%%%%%%%%%%%%%%%%%%%%%%%%%%%%%%%%%%%%%%%%%%%%%%%%%%%%%
%%%%%%%%%%%%%%%%%%%%%%%%%%%%%%%%%%%%%%%%%%%%%%%%%%%%%
%%%%% Trace names

\newcommand{\pit}{\pi\text{-trace}}
\newcommand{\taut}{\tau\text{-trace}}
\newcommand{\pits}{\pi\text{-traces}}
\newcommand{\tauts}{\tau\text{-traces}}

%%%%%%%%%%%%%%%%%%%%%%%%%%%%%%%%%%%%%%%%%%%%%%%%%%%%%
%%%%%%%%%%%%%%%%%%%%%%%%%%%%%%%%%%%%%%%%%%%%%%%%%%%%%
%%%%% Complexity

\newcommand{\poly}{\textsc{P}\xspace}
\newcommand{\np}{\textsc{NP}\xspace}
\newcommand{\conp}{\textsc{coNP}\xspace}
\newcommand{\sigmatwo}{$\Sigma_2^{\textsc{P}}$\xspace}
\newcommand{\pitwo}{$\Pi_2^{\textsc{P}}$\xspace}
\newcommand{\sigmathree}{$\Sigma_3^{\textsc{P}}$\xspace}
\newcommand{\pspace}{\textsc{PSpace}\xspace}
\newcommand{\expt}{\textsc{ExpTime}\xspace}
\newcommand{\nexpt}{\textsc{NExpTime}\xspace}
\newcommand{\exps}{\textsc{ExpSpace}\xspace}
\newcommand{\twoexpt}{\textsc{2ExpTime}\xspace}
\newcommand{\tower}{\textsc{Tower}\xspace}

%%%%%%%%%%%%%%%%%%%%%%%%%%%%%%%%%%%%%%%%%%%%%%%%%%%%%
%%%%%%%%%%%%%%%%%%%%%%%%%%%%%%%%%%%%%%%%%%%%%%%%%%%%%
%%%%% Trees
\newcommand{\aut}{\mathcal{A}}
\newcommand{\lang}[1]{\mathcal{L}(#1)}
\newcommand{\initmark}{I}
\newcommand{\kripke}{\mathcal{K}}
\newcommand{\src}[1]{src(#1)}
\newcommand{\tgt}[1]{tgt(#1)}
\newcommand{\runs}[1]{\mathcal{R}(#1)}
\newcommand{\traces}[1]{\mathcal{L}(#1)}
\newcommand{\trace}[1]{tr(#1)}
\newcommand{\game}{\mathcal{G}}
\newcommand{\val}[2]{val_{#1,#2}}
\newcommand{\valstrat}[3]{val^{#2,#3}_{#1}}
\newcommand{\trees}[1]{\mathcal{T}(#1)}
\newcommand{\subformulas}[1]{{Sub(#1)}}
\newcommand{\sons}[2]{Sons_{#2}(#1)}
\newcommand{\treeroot}[1]{root(#1)}
\newcommand{\ancestor}[1]{\preccurlyeq_{#1}}
\newcommand{\descendant}{\succcurlyeq}

%%%%%%%%%%%%%%%%%%%%%%%%%%%%%%%%%%%%%%%%%%%%%%%%%%%%%
%%%%%%%%%%%%%%%%%%%%%%%%%%%%%%%%%%%%%%%%%%%%%%%%%%%%%
%%%%% Probabilities

\newcommand{\distri}[1]{\mathcal{D}(#1)}
\newcommand{\proba}[1]{\mathbb{P}(#1)}

\newcommand{\corto}[1]{\todo[color=red!30,inline]{\small #1}}




\title{Undecidability of verification of $\set{0,1, \omega}$ strategies}
\author{}
\date{}

\begin{document}
	
	\maketitle

Consider a two-counter system $\S = (S, \delta, init, fin)$ with $\delta : S \times \set{\top, \bot}^2 \to \set{1,2} \times Op \times S$ and $Op = \set{++, --}$.
Intuitively, $\delta$ takes as input the current state and two booleans indicating if the counters are equal to $0$ or not. It outputs an operation, a counter on which to apply it and a new state.

Note that this machine has a single run from its initial configuration $(init, 0, 0)$.
The problem whether such a machine is unbounded (that is, one of its counters reaches arbitrarily high values along its run) is undecidable.

Consider the following population game: There is a place for each state $s \in S$, two places for each counter, two more forming a reservoir of tokens, plus a starting and a final place.

We have an action for each $(s, b1, b2) \in S \times \set{\top, \bot}^2$, plus special letters $start, empty, win, shuffle$.

The letter $start$ takes all tokens from the starting place and splits them between the reservoir and the place associated to $init$.
The letter $empty$ can only be played if all places associated with states are empty, and it sends all tokens to the final place.
As for $win$, it can only be played if the reservoir is empty, and it sends all tokens to the final place.
Finally, $shuffle$ redistributes randomly the tokens between each pair of states associated with the reservoir and each counter.

For each $(s, b1, b2)$, with $\delta(s, b1, b2) = (i, op, t)$, the associated action can only be played if all places associated with states besides $s$ are empty, and transfers the ones in $s$ to $t$.

If $op = ++$ then it transfers all tokens from one of the two places of the reservoir to counter $i$.
If $op = --$ then it transfers all tokens from one of the two places of counter $i$ to the reservoir.

We consider the strategy that consists in executing faithfully the two-counter machine. It starts by executing $start$. If no token reaches the places associated with states, it reads $empty$ and wins.

Otherwise it executes $shuffle$ until in the three pairs of places associated with the counters and the reservoir the tokens are split between one token on a side and all others on the other side.

Then it reads the action associated with $s, b1, b2$, where $s$ is the unique state where there are tokens, and $b1$ and $b2$ are true if, respectively, there are no tokens in the states of counters $1$ and $2$. Thanks to the previous paragraph, we know that exactly one token is transferred from the decremented counter to the reservoir, or from the reservoir to the incremented counter.

If at any point the reservoir is empty, it plays action $win$.

An easy induction shows that if the play yielded by this strategy goes through $n$ actions $(s,b1, b2)$, then those are the $n$ first actions of the machine, and if the configuration of the machine after executing them is $(s, c1, c2)$, then the the resulting configuration in the game is such that 
\begin{itemize}
	\item There are tokens in $s$ but not in any other state $t$.
	
	\item There are $c1$ tokens in total in the two places of counter $1$, and $c2$ in the places of counter $2$.
\end{itemize}

As a result if the machine is bounded by a bound $B$, then if at the start there are $2B + 2$ tokens, there is a positive probability that one of them is sent to the states, and $2B+1$ to the reservoir at the start. Then the reservoir will never be empty (as there are at most $2B$ tokens in the counters and one in the states and zero elsewhere at all times), and neither will the states. Hence the strategy is not winning.

If the machine is unbounded, then let $N \in \nats$, say we start with $N$ tokens.
If none of them are sent to the states then we win immediately by playing $empty$.
Otherwise there are $K < N$ tokens sent to the reservoir. The strategy simulates the machine until we reach a configuration where the sum of the counters is above $K$ (which happens eventually as it is unbounded). Then we play $win$ and win.

As a result, the strategy is winning if and only if the machine is unbounded.
Furthermore this strategy bases its decisions solely on the $0, 1, \omega$ abstraction of the current configuration.

As a consequence, the verification of a (positional) $(0, 1, \omega)$ strategy for population games is undecidable.

	
\end{document}